% Folie 1: Motivation Zeitabhängigkeit
\begin{frame}{Von stationär zu dynamisch}
    \textbf{Limit der zeitunabhängigen SG:}
    \begin{itemize}
        \item Beschreibt nur \alert{stationäre Zustände} ($|\psi|^2$ zeitkonstant)
        \item Beispiel: Elektron im Atom (Grundzustand)
    \end{itemize}

    \textbf{Probleme mit Bewegung:}
    \begin{itemize}
        \item Wie beschreibt man Teilchen, die sich bewegen? (z.B. Elektronenstrahl)
        \item Was passiert bei zeitabhängigen Potenzialen $V(x,t)$?
    \end{itemize}

    \centering
    \alert{$\Rightarrow$ Brauch eine Gleichung für $\psi(x,t)$!}
\end{frame}

% Folie 2: Herleitung der TDSE
\begin{frame}{Herleitung der zeitabhängigen SG}
    \textbf{Ansatz:} Energie-Operator $\hat{E} = i\hbar\frac{\partial}{\partial t}$\\
    Analog zu $\hat{p} = -i\hbar\frac{\partial}{\partial x}$

    \textbf{Energiegleichung:}
    \[
        \underbrace{\hat{H}}_{\text{Hamiltonoperator}} \psi = \underbrace{\hat{E}}_{\text{Energieoperator}} \psi
    \]

    \textbf{Einsetzen der Operatoren:}
    \[
        -\frac{\hbar^2}{2m}\frac{\partial^2 \psi}{\partial x^2} + V(x)\psi = i\hbar\frac{\partial \psi}{\partial t}
    \]

    \textbf{Allgemeine Form:}
    \[
        \boxed{\hat{H}\psi(x,t) = i\hbar\frac{\partial}{\partial t}\psi(x,t)}
    \]
\end{frame}

% Folie 3: Vergleich Zeitabhängig vs. Unabhängig
\begin{frame}{Vergleich der Gleichungen}
    \begin{columns}
        \column{0.5\textwidth}
        \textbf{Zeitunabhängig:}
        \[
            \hat{H}\psi = E\psi
        \]
        \begin{itemize}
            \item Eigenwertgleichung
            \item Lösungen: $\psi_n(x)e^{-iE_n t/\hbar}$
            \item Stationäre Zustände
        \end{itemize}

        \column{0.5\textwidth}
        \textbf{Zeitabhängig:}
        \[
            \hat{H}\psi = i\hbar\dot{\psi}
        \]
        \begin{itemize}
            \item Differentialgleichung 1. Ordnung in $t$
            \item Beschreibt Zeitentwicklung
            \item Allgemeine Lösung: Überlagerung von $\psi_n$
        \end{itemize}
    \end{columns}
\end{frame}

% Folie 4: Lösungsansatz (Separation)
\begin{frame}{Lösungsmethode: Separation der Variablen}
    \textbf{Ansatz:} $\psi(x,t) = \phi(x)T(t)$\\
    Einsetzen in TDSE:
    \[
        \frac{\hat{H}\phi(x)}{\phi(x)} = \frac{i\hbar \dot{T}(t)}{T(t)} = E
    \]

    \textbf{Zwei Gleichungen:}
    \begin{enumerate}
        \item Zeitunabhängige SG: $\hat{H}\phi = E\phi$
        \item Zeitentwicklung: $T(t) = e^{-iEt/\hbar}$
    \end{enumerate}

    \textbf{Vollständige Lösung:}
    \[
        \psi(x,t) = \phi(x)e^{-iEt/\hbar}
    \]
    \begin{itemize}
        \item Phasenfaktor $e^{-iEt/\hbar}$ ändert $|\psi|^2$ nicht
    \end{itemize}
\end{frame}

% Folie 5: Wellenpakete (Vorbereitung Doppelspalt)
\begin{frame}{Wellenpaket-Animation (1D-Zeitentwicklung)}
    \centering
    \begin{animateinline}[
        controls={play, step, stop},
        loop,
        poster=first,
        width=0.8\textwidth
    ]{10} % 10 Frames pro Sekunde
        \multiframe{20}{rt=0.0+0.25} % 20 Frames von t=0 bis t=5
        {
            \begin{tikzpicture}
                \begin{axis}[
                    width=0.8\textwidth,
                    height=6cm,
                    xlabel=$x$,
%                        ylabel=$|\psi(x,t)|^2$,
                    ylabel=$|\psi (x\,t)|^2$,
                    xmin=-3, xmax=3,
                    ymin=0, ymax=1.2,
                    grid=major
                ]
                    % Hüllkurve
                    \addplot[red!50, thick, domain=-3:3, samples=50] {
                        exp(-(x - 0.5*\rt)^2)
                    };

                    % Wellenpaket mit Oszillationen
                    \addplot[blue, thick, domain=-3:3, samples=100] {
                        exp(-(x - 0.5*\rt)^2) * (1 + 0.3*cos(deg(5*x - 2*\rt)))
                    };

%                         Zeitmarker
                    \node[draw] at (axis cs:2,1) {$t = \pgfmathprintnumber{\rt}$};
                \end{axis}
            \end{tikzpicture}
        }
    \end{animateinline}

%        \begin{itemize}
%            \item Klicken Sie auf "Play" um die Zeitentwicklung zu starten
%            \item Geschwindigkeit mit "+/-" anpassbar
%            \item Deutlich sichtbar: Bewegung + Dispersion + Interferenz
%        \end{itemize}
\end{frame}

\begin{frame}{Wellenpaket-Animation (2D-Zeitentwicklung)}
    \centering
    \begin{animateinline}[
        controls={play, step, stop},
        loop,
        poster=first,
        width=0.8\textwidth
    ]{10} % 10 Frames pro Sekunde
        \multiframe{20}{rt=0.0+0.25} % 20 Frames von t=0 bis t=5
        {
            \centering
            \begin{tikzpicture}
                \begin{axis}
                    [
                    width=0.8\textwidth,
                    height=6cm,
                    colormap/viridis,
                    view = {45}{30},
                    xlabel = $x$,
                    ylabel = $y$,
                    axis line style={draw=none},
                    zlabel = {Wahrscheinlichkeitsdichte $|\psi(x,y)|^2$},
                    label style = {sloped},
                    grid = major,
                    grid style = {dashed, gray!30},
                    tick label style = {font=\scriptsize},
                    title = {Moduliertes Gaußsches Wellenpaket},
                    title style = {font=\small},
                    colorbar,
                    colorbar style={
                        title=Intensität,
                        yticklabel style={/pgf/number format/.cd,fixed,precision=2}
                    }
                    ]

                    % Hüllkurve
                    \addplot3[
                    surf,
                    shader = interp,
                    samples = 20,
                    domain = -3:3,
                    y domain = -3:3,
%                point meta=abs,
                    ] {
%                    exp(-(x^2 + y^2))       % Gaußscher Kern
%                    * (1 + 0.4*cos(6*x))    % Wellenmodulation
%                    * (1 + 0.4*cos(6*y))    % 2D-Interferenzmuster
%                    exp(-(x^2 + y^2))* (1 + 0.4*cos(6*x))* (1 + 0.4*cos(6*y))
%                        (1 + 0.4*cos(6* deg(x)))* (1 + 0.4*cos(6*deg(y))) * exp(-(x^2 + y^2))
                        exp(-((x - 0.5*\rt)^2 + (y)^2))
                    };


%                    (1 + 0.4*cos(6* deg(x)))* (1 + 0.4*cos(6*deg(y))) * exp(-(x^2 + y^2))
%                    % Hüllkurve
%                    \addplot[red!50, thick, domain=-3:3, samples=50] {
%                        exp(-(x - 0.5*\rt)^2)
%                    };

%                    % Wellenpaket mit Oszillationen
%                    \addplot[blue, thick, domain=-3:3, samples=100] {
%                        exp(-((x - 0.5*\rt)^2)) * (1 + 0.3*cos(deg(5*x - 2*\rt)))
%                    };

%                    \node[white,font=\tiny] at (axis cs:-2.5,-2.5,0.7)
%                        {$\psi(x,y) = e^{-(x^2+y^2)}(1 + 0.4\cos 6x)(1 + 0.4\cos 6y)$};
                \end{axis}
%            \end{tikzpicture}
%            \begin{tikzpicture}
%                \begin{axis}[
%                    width=0.8\textwidth,
%                    height=6cm,
%                    xlabel=$x$,
%%                        ylabel=$|\psi(x,t)|^2$,
%                    ylabel=$|\psi (x\,t)|^2$,
%                    xmin=-3, xmax=3,
%                    ymin=0, ymax=1.2,
%                    grid=major
%                ]
%
%%                    (1 + 0.4*cos(6* deg(x)))* (1 + 0.4*cos(6*deg(y))) * exp(-(x^2 + y^2))
%                    % Hüllkurve
%                    \addplot[red!50, thick, domain=-3:3, samples=50] {
%                        exp(-(x - 0.5*\rt)^2)
%                    };
%
%                    % Wellenpaket mit Oszillationen
%                    \addplot[blue, thick, domain=-3:3, samples=100] {
%                        exp(-((x - 0.5*\rt)^2)) * (1 + 0.3*cos(deg(5*x - 2*\rt)))
%                    };
%
%%                         Zeitmarker
%                    \node[draw] at (axis cs:2,1) {$t = \pgfmathprintnumber{\rt}$};
%                \end{axis}
            \end{tikzpicture}
        }
    \end{animateinline}

%        \begin{itemize}
%            \item Klicken Sie auf "Play" um die Zeitentwicklung zu starten
%            \item Geschwindigkeit mit "+/-" anpassbar
%            \item Deutlich sichtbar: Bewegung + Dispersion + Interferenz
%        \end{itemize}
\end{frame}
% Folie 6: Doppelspalt-Experiment vorbereiten
\begin{frame}{Der Weg zum Doppelspalt}
    \textbf{Modellierung im 2D:}
    \[
        i\hbar\frac{\partial \psi}{\partial t} = \left(-\frac{\hbar^2}{2m}(\frac{\partial^2}{\partial x^2} + \frac{\partial^2}{\partial y^2}) + V(x,y)\right)\psi
    \]

    \textbf{Potenzial beim Doppelspalt:}
    \[
        V(x,y) = \begin{cases}
                     \infty & \text{(Wand)} \\
                     0 & \text{(Spalte)}
        \end{cases}
    \]

    \textbf{Lösungsstrategie:}
    \begin{enumerate}
        \item Anfangsbedingung: Wellenpaket vor der Wand
        \item Zeitentwicklung berechnen (numerisch)
        \item Interferenzmuster entsteht durch $\psi_1 + \psi_2$
    \end{enumerate}
\end{frame}


\begin{frame}{Doppelspaltexperiment – Live-Animation}
    \centering
%\movie[width=0.8\textwidth,height=0.6\textwidth,poster,showcontrols]{}{../double_slit.gif}

%    \animategraphics[loop,controls,width=\linewidth]{10}{./render/frame_}{0000}{0199}


%\begin{itemize}
%    \item Wellenausbreitung durch beide Spalte
%    \item Interferenzmuster entsteht am Detektor
%    \item Typisches Quantenphänomen: Selbstinterferenz
%\end{itemize}
\end{frame}
