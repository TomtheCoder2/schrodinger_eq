\begin{frame}{Wellenpaket-Animation (1D-Zeitentwicklung)}
    \centering
    \begin{animateinline}[
        controls={play, step, stop},
        loop,
        poster=first,
        width=0.8\textwidth
    ]{10} % 10 Frames pro Sekunde
        \multiframe{20}{rt=0.0+0.25} % 20 Frames von t=0 bis t=5
        {
            \begin{tikzpicture}
                \begin{axis}[
                    width=0.8\textwidth,
                    height=6cm,
                    xlabel=$x$,
%                        ylabel=$|\psi(x,t)|^2$,
                    ylabel=$|\psi (x\,t)|^2$,
                    xmin=-3, xmax=3,
                    ymin=0, ymax=1.2,
                    grid=major
                ]
                    % Hüllkurve
                    \addplot[red!50, thick, domain=-3:3, samples=50] {
                        exp(-(x - 0.5*\rt)^2)
                    };

                    % Wellenpaket mit Oszillationen
                    \addplot[blue, thick, domain=-3:3, samples=100] {
                        exp(-(x - 0.5*\rt)^2) * (1 + 0.3*cos(deg(5*x - 2*\rt)))
                    };

%                         Zeitmarker
                    \node[draw] at (axis cs:2,1) {$t = \pgfmathprintnumber{\rt}$};
                \end{axis}
            \end{tikzpicture}
        }
    \end{animateinline}

%        \begin{itemize}
%            \item Klicken Sie auf "Play" um die Zeitentwicklung zu starten
%            \item Geschwindigkeit mit "+/-" anpassbar
%            \item Deutlich sichtbar: Bewegung + Dispersion + Interferenz
%        \end{itemize}
\end{frame}

\begin{frame}{Wellenpaket-Animation (2D-Zeitentwicklung)}
    \centering
    \begin{animateinline}[
        controls={play, step, stop},
        loop,
        poster=first,
        width=0.8\textwidth
    ]{10} % 10 Frames pro Sekunde
        \multiframe{20}{rt=0.0+0.25} % 20 Frames von t=0 bis t=5
        {
            \centering
            \begin{tikzpicture}
                \begin{axis}
                    [
                    width=0.8\textwidth,
                    height=6cm,
                    colormap/viridis,
                    view = {45}{30},
                    xlabel = $x$,
                    ylabel = $y$,
                    axis line style={draw=none},
                    zlabel = {Wahrscheinlichkeitsdichte $|\psi(x,y)|^2$},
                    label style = {sloped},
                    grid = major,
                    grid style = {dashed, gray!30},
                    tick label style = {font=\scriptsize},
                    title = {Moduliertes Gaußsches Wellenpaket},
                    title style = {font=\small},
                    colorbar,
                    colorbar style={
                        title=Intensität,
                        yticklabel style={/pgf/number format/.cd,fixed,precision=2}
                    }
                    ]

                    % Hüllkurve
                    \addplot3[
                    surf,
                    shader = interp,
                    samples = 20,
                    domain = -3:3,
                    y domain = -3:3,
%                point meta=abs,
                    ] {
%                    exp(-(x^2 + y^2))       % Gaußscher Kern
%                    * (1 + 0.4*cos(6*x))    % Wellenmodulation
%                    * (1 + 0.4*cos(6*y))    % 2D-Interferenzmuster
%                    exp(-(x^2 + y^2))* (1 + 0.4*cos(6*x))* (1 + 0.4*cos(6*y))
%                        (1 + 0.4*cos(6* deg(x)))* (1 + 0.4*cos(6*deg(y))) * exp(-(x^2 + y^2))
                        exp(-((x - 0.5*\rt)^2 + (y)^2))
                    };


%                    (1 + 0.4*cos(6* deg(x)))* (1 + 0.4*cos(6*deg(y))) * exp(-(x^2 + y^2))
%                    % Hüllkurve
%                    \addplot[red!50, thick, domain=-3:3, samples=50] {
%                        exp(-(x - 0.5*\rt)^2)
%                    };

%                    % Wellenpaket mit Oszillationen
%                    \addplot[blue, thick, domain=-3:3, samples=100] {
%                        exp(-((x - 0.5*\rt)^2)) * (1 + 0.3*cos(deg(5*x - 2*\rt)))
%                    };

%                    \node[white,font=\tiny] at (axis cs:-2.5,-2.5,0.7)
%                        {$\psi(x,y) = e^{-(x^2+y^2)}(1 + 0.4\cos 6x)(1 + 0.4\cos 6y)$};
                \end{axis}
%            \end{tikzpicture}
%            \begin{tikzpicture}
%                \begin{axis}[
%                    width=0.8\textwidth,
%                    height=6cm,
%                    xlabel=$x$,
%%                        ylabel=$|\psi(x,t)|^2$,
%                    ylabel=$|\psi (x\,t)|^2$,
%                    xmin=-3, xmax=3,
%                    ymin=0, ymax=1.2,
%                    grid=major
%                ]
%
%%                    (1 + 0.4*cos(6* deg(x)))* (1 + 0.4*cos(6*deg(y))) * exp(-(x^2 + y^2))
%                    % Hüllkurve
%                    \addplot[red!50, thick, domain=-3:3, samples=50] {
%                        exp(-(x - 0.5*\rt)^2)
%                    };
%
%                    % Wellenpaket mit Oszillationen
%                    \addplot[blue, thick, domain=-3:3, samples=100] {
%                        exp(-((x - 0.5*\rt)^2)) * (1 + 0.3*cos(deg(5*x - 2*\rt)))
%                    };
%
%%                         Zeitmarker
%                    \node[draw] at (axis cs:2,1) {$t = \pgfmathprintnumber{\rt}$};
%                \end{axis}
            \end{tikzpicture}
        }
    \end{animateinline}

%        \begin{itemize}
%            \item Klicken Sie auf "Play" um die Zeitentwicklung zu starten
%            \item Geschwindigkeit mit "+/-" anpassbar
%            \item Deutlich sichtbar: Bewegung + Dispersion + Interferenz
%        \end{itemize}
\end{frame}